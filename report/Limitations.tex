\chapter{Limitations}
As mentioned in \autoref{chapter:results} the type analyser doesn't support the entire language; some features are partially supported and others are not supported at all. In this chapter we will reiterate the limitations for the project.

Being a full fledged general-purpose programming language, Python has many different features. In our analysis we have decided to exclude some of these as they don't add anything when taking our goals into account;

\begin{itemize}
	\item Lambda expressions
	\item Generator expressions
	\item Yield expressions
    \item Exec and eval expressions
\end{itemize}

Especially generator and yield expressions present a great deal of complexity\footnote{Shriram and Joe et. al. support these features and in their paper they describe their results while investigating these features \cite{lambdapy}.} and remain unhandled. Other limitations include the following:

\begin{description}
	\item[Magic methods] With the focus being on supporting \inlinecode{\_\_getattr\_\_} there has been given little attention to other magic methods. By completely omitting \inlinecode{\_\_getattribute\_\_} we noted in \autoref{Magic methods transformation} that it becomes easier to support the magic method \inlinecode{\_\_getattr\_\_}. A complete list and description of the magic methods can be found in The Python Language Reference data model \cite{pyref.datamodel} in section 3.4.
	\item[Exceptions] As a means to an end exceptions are partially supported. In particular there is only support for catch-all except blocks, i.e. except blocks that does not specify a type.
	\item[Standard Library] A lot of the functionality in Python comes from the Python Standard Library which includes built-in constants, functions and classes. Built-in structures can be interpreted as being implicitly imported.
	\item[Imports] Unlike JavaScript Python supports modules, which are used in any large Python project.
\end{description}