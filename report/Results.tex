\chapter{Results}
During our project we have developed a type analyzer for Python, which is able to analyze simple Python programs. In this section we present some small, but non-trivial to analyze, programs together with the results of our type analyzer. We aimed to be able to support simple use of the magic method \inlinecode{\_\_getattr\_\_}, as we found that all uses of it in the web framework Django were simple.

In order to do so we needed to handle among others simple use of exceptions. Consider the following calculator example that makes use of exceptions:

\begin{listing}[H]
	\begin{minted}[linenos]{python}
def calculator(a, op, b):
	if (op == "+"):
		result = a + b
	elif (op == "-"):
		result = a - b
	elif (op == "*"):
		result = a * b
	elif (op == "/"):
		result = a / b
	else:
		raise Exception()
	return result

try:
	amodb  = calculator(10, "%", 20)
except:
	err = "An error occured"
	\end{minted}
\end{listing}

For this example our analyzer will conclude that \inlinecode{amodb} is undefined, and that \inlinecode{err} is "An error occurred". Due to a very simple path sensitivity our analyzer does not conclude that \inlinecode{amodb} is either undefined, \inlinecode{a}+\inlinecode{b}, \inlinecode{a}-\inlinecode{b}, \inlinecode{a}*\inlinecode{b} or \inlinecode{a}/\inlinecode{b}. If we had instead called \inlinecode{calculator(10, "+", 20)} our analyzer would correctly conclude that the result of the function call would be 30.

The simple exception handling has allowed us to handle implicit \inlinecode{\_\_getattr\_\_} calls. Consider part of the \inlinecode{Student} example from section \ref{Features} about dynamic features that uses the magic method \inlinecode{\_\_getattr\_\_}:

\begin{listing}[H]
	\begin{minted}[linenos]{python}
class Student(object):
  def __init__(self, name):
    self.name = name
  def __getattr__(self, name):
    if name in self.grades:
      return self.grades[name]
    else:
      raise AttributeError()
a = Student('John')
a.grades = { 'math': 'A' }
try:
	mathgrade = a.math
except:
	err = "Error"
	\end{minted}
\end{listing}

Our tool is able to analyze this program and conclude that \inlinecode{mathgrade} is either 'A' or undefined (the latter because we do a weak update in line 10 to \inlinecode{a.grades}). It should be mentioned though, that the lack of context sensivity destroys the precision very quickly because Python has a lot of implicit method calls, contrary to e.g. JavaScript.

Again however, it is important to stress that the fact that we only support a subset of Python, e.g. not the magic method \inlinecode{\_\_getattribute\_\_}, simplifies the situation as also mentioned in the section about magic methods, \ref{Magic methods transformation}.
