\chapter{Introduction}
Python is a dynamically typed, general purpose programming language that supports both object-oriented, imperative and functional programming styles. Because of its dynamic nature it has a lot of similar features to JavaScript, which makes it difficult to statically analyze. This is further complicated by langauges features supported in Python which is non-trivial to analyse. An example of such an non-trivial language feature is Pythons magic methods, which results in implicit method and function calls; this makes it harder to predict the outcome of otherwise relatively simple statements and expressions, as e.g. an attribute lookup.

Python is widely used in both education and industry, and because of this popularity IDEs\cite{ide.appcelerator, ide.jetbrains, ide.wingware} and other third-party tools\cite{tool.pep8, tool.pyflakes, tool.pychecker, tool.pylint} has been developed to accomondate the developer by finding errors and making refacorings. Sadly all of these tools is affected by the complexity in the Python language and thereby shown to be unsound\cite{lamdapy}. In this report we present our work towards developering a conservaite and sound type analysis for Python version 2.7.5\footnote{This version is still predominantly used in the wild} written in Scala, furthermore we uses the third-party project Jython\cite{jython} to parse Python.

\section{Motivation and Contributions}
Inspired by the Type Analysis for JavaScript\cite{tajs} or aim for this project is to develop a type analysis for Python which will be able to analyze simple Python programs that use some of the non-trivial features, like magic methods, that distinct Python from other dynamically typed languages as e.g. JavaScript. Especially, we wish to be able to do type analysis on programs that uses the magic method \inlinecode{\_\_getattr\_\_}, which is called whenever an attribute lookup results in an \inlinecode{AttributeError}. As this should already indicate, our type analyzer should therefore also be able to handle among others declaration and instantiation of classes together with exceptions.

This paper presents a proof of concept that static analysis of Python is possible using the traditional Monotone framework. In the paper we give Control Flow Graph (CFG) constructions for varius statements and expressions for Python including a dynamic way of changing the CFG during evaluation of the analysis to accomondate the support for the magic method \inlinecode{\_\_getattr\_\_}.  

During the development of this project we have identified some of the intriguing features in Python. This includes python scope rules and the order of evaluation when handling magic methods.